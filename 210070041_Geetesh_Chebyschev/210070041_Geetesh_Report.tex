\documentclass{article}
\usepackage{graphicx} % Required for inserting images
\usepackage{amsmath}
\usepackage{float}

\title{Chebyschev Filter Design Report}
\author{Name: Geetesh Kini, Roll no: 210070041, Filter no: 104, Group No: 15,\\ Reviewed by Group member: Chanakya Varude, Roll no: 210070092}
\date{March 2023}

\begin{document}

\maketitle

\tableofcontents
\clearpage

\section{Filter Details (Bandpass)}

\subsection{Un-normalized Discrete Time Filter Specifications}

Filter number = 104\\
Since filter number $>$ 80, m = 104 - 80 = 24 and passband will be equiripple.\\
q(m) = greatest integer strictly less than 0.1*m = 2\\
r(m) = m - 10*q(m) = 4\\
$B_L$(m) = 10 + 5*q(m) + 13*r(m) = 10 + 5*2 + 13*4 = 72KHz \\
$B_H$(m) = $B_L$(m) + 75 = 147KHz\\

\vspace{1.5em}
\noindent

This filter is given to be a Bandpass Filter with passband from BL(m) kHz to BH(m) kHz.
Therefore the specifications are:
\begin{itemize}
    \item Passband : \textbf{72 - 147 KHz}
    \item  Transition band : \textbf{5KHz} on either side of stopband
    \item Stopband : \textbf{0 - 67} and \textbf{152 - 300 KHz} (As \textbf{sampling rate} is \textbf{600KHz})

    \item  Tolerance : \textbf{0.15} in \textbf{magnitude} for both passband and stopband
    \item  Nature : Passband is \textbf{equiripple} and stopband is \textbf{monotonic}
\end{itemize}

\subsection{Normalized Digital Filter Specifications}
Sampling rate = 600KHz\\
In the normalized frequency axis, sampling rate corresponds to 2$\pi$\\
Therefore, any frequency can be normalized as follows :
\begin{equation*}
    \omega = \frac{\Omega*2\pi}{\Omega_s}
\end{equation*}
where $\Omega_s$ is the Sampling Rate.\\

\vspace{1em}
\noindent
For the normalized discrete filter specifications, the nature and tolerances being the dependent variables remain the same while the passband and stopband frequencies change as per the above transformations. 
\begin{itemize}
    \item Passband : (\textbf{0.24 -  0.49}) {$\pi$}
    \item  Transition band : \textbf{0.017} $\pi$ on either side of stopband
    \item Stopband : (\textbf{0 - 0.223}) {$\pi$} and (\textbf{0.507 - 1}) {$\pi$}
    \item  Tolerance : \textbf{0.15} in \textbf{magnitude} for both passband and stopband
    \item Passband nature: \textbf{Equiripple}
    \item Stopband nature: \textbf{Monotonic}
\end{itemize}


\subsection{Band-pass analog Filter Specifications using Bilinear
Transformation}
To convert to analog domain, we use the following transformation :
\begin{equation*}
    \Omega = \tan (\frac{w}{2})
\end{equation*}
Applying the transformation at Band Edges we get :
\begin{table}[H]
		% Center the table
		\begin{center}
		\begin{tabular}{|c|c|}
			% To create a horizontal line, type \hline
			\hline
			% To end a column type &
			% For a linebreak type \\
			$\omega$ & $\Omega$\\
			
			\hline
                0 & 0\\
                \hline
                0.24 $\pi$ & 0.396 \\
                \hline
                0.49 $\pi$ & 0.969\\
                \hline
                0.223 $\pi$ & 0.365\\
                \hline
                0.507 $\pi$ & 1.022\\
                \hline
                $\pi$ & $\infty$\\
                \hline
            
		\end{tabular}
		\end{center}
\end{table}

Hence, the specifications for corresponding bandpass analog filter are:
\begin{itemize}
    \item Passband : (\textbf{0.396 -  0.969})
    \item  Transition band : \textbf{0.365 to 0.396} and \textbf{0.969 to 1.022}
    \item Stopband : (\textbf{0 - 0.365}) and (\textbf{1.022 - $\infty$})
    \item  Tolerance : \textbf{0.15} in \textbf{magnitude} for both passband and stopband
    \item Passband nature: \textbf{Equiripple}
    \item Stopband nature: \textbf{Monotonic}
\end{itemize}

\subsection{Low pass analog Filter Specifications using Frequency
Transformation}

We need to transform the bandpass filter to a low pass filter. Hence, we can use the bandpass transformation:
\vspace{-5mm}
\begin{center}
    \begin{equation*}
        \Omega_L = \frac{\Omega^2-\Omega_0^2}{B\Omega}
    \end{equation*}
\end{center}

Where $\Omega_0$ and B can be determined by:
We want $\Omega_{P1}$ to map to -1 and $\Omega_{P2}$ to map to +1.\\
Using this, we get:
\begin{center}
    \begin{equation*}
        \Omega_0 = \sqrt{\Omega_{P1}\Omega_{P2}} = \sqrt{0.396 \times 0.969} = 0.61945
    \end{equation*}
    \begin{equation*}
        B = \Omega_{P2} - \Omega_{P1} = 0.969 - 0.396 = 0.573
    \end{equation*}
\end{center}

\begin{table}[H]
		% Center the table
		\begin{center}
		\begin{tabular}{|c|c|}
			% To create a horizontal line, type \hline
			\hline
			% To end a column type &
			% For a linebreak type \\
			$\Omega$ & $\Omega_L$\\
			
			\hline
                $0^+$ & - $\infty$\\
                \hline
                0.365($\Omega_{S1}$) & -1.198$\Omega_{L_{S1}}$)\\
                \hline
                0.396($\Omega_{P1}$) & -1($\Omega_{L_{P1}}$)\\
                \hline
                0.61945($\Omega_0$) & 0\\
                \hline
                0.969($\Omega_{P2}$) & +1($\Omega_{L_{P2}}$) \\
                \hline
                1.022($\Omega_{S2}$) & 1.128$\Omega_{L_{S2}}$)\\
                \hline
                $\infty$ & $\infty$\\
                \hline
            
		\end{tabular}
		\end{center}
\end{table}

\subsection{Low pass Analog filter specifications}

\begin{itemize}
    \item Passband Edge : 1 ($\Omega_{L_{P}}$)
    \item Stopband Edge : min(-$\Omega_{L_{S1}}$, $\Omega_{L_{S2}}$) = min(1.198, 1.128) = 1.128 ($\Omega_{L_{S}}$)
    \item  Tolerance : \textbf{0.15} in \textbf{magnitude} for both passband and stopband
    \item Passband nature: \textbf{Equiripple}
    \item Stopband nature: \textbf{Monotonic}
\end{itemize}

\subsection{Analog Lowpass Transfer Function}

The analog filter has Equiripple passband and Monotonic stopband, hence, we need a \textbf{Chebyschev} filter. Now, using the given tolerance as 0.15, we can define 2 new quantities $D_1$ and $D_2$:

\begin{center}
    \begin{equation*}
        D_1 = \frac{1}{(1-\delta)^2} - 1 = \frac{1}{0.85^2} - 1 = 0.3841
    \end{equation*}
    \begin{equation*}
        D_2 = \frac{1}{\delta^2} - 1 = \frac{1}{0.15^2} - 1 = 43.44
    \end{equation*}
\end{center}

Hence, the order of the filter N satisfies the inequality:
\begin{center}
    \begin{equation*}
        N \geq \frac{cosh^{-1}(\sqrt{\frac{D_2}{D_1}})}{cosh^{-1}(\frac{\Omega_S}{\Omega_P})} = 6.101
    \end{equation*}
\end{center}

And we want minimum resources to design this filter, hence we use the smallest possible integer value of n that satisfies the above equation, that is \textbf{N = 7}. Also, we choose $\epsilon$ to be $\sqrt{D_1}$.\\

Now, the poles are the roots of the equation:

\begin{center}
    \begin{equation*}
        1+D_1cosh^2(N_{min}cosh^{-1}(\frac{s}{j}))=1+0.3841cosh^2(4cosh^{-1}(\frac{s}{j}))=0
    \end{equation*}
\end{center}

Solving for roots, using Wolfram, we get:

\begin{figure}[h!]

\centering
\includegraphics[scale = 0.8]{Wol_Cheb.png}
\caption{Poles of Magnitude Plot of Analog LPF}
\end{figure}

For a stable filter, we need to take only the LHP poles, so:\\

$p_1$ = -0.18041\\
$p_2$ = -0.11249 + 0.79445 i\\
$p_3$ = -0.11249 - 0.79445 i\\
$p_4$ = -0.04015 + 0.99067 i\\
$p_5$ = -0.04015 - 0.99067 i\\
$p_6$ = -0.16255 + 0.44089 i\\
$p_7$ = -0.16255 - 0.44089 i\\

Using the above poles which are in the left half plane we can write the Analog Lowpass Transfer Function as:

      $   H_{analog,LPF}(s_L) = \frac{(-1)^N\Pi_{i=1}^{N} p_i}{\Pi_{i=1}^{N} (s_L-p_i)} $\\
     $ = \frac{0.0252}{(s_L+0.18)(s_L^2+0.225s_L+0.644)(s_L^2+0.08s_L+0.983)(s_L^2+0.325s_L+0.221)} $\\


    Note that since it is odd order we take the DC Gain to be 1.


\subsection{Analog Bandpass Transfer Function}

The transformation is given by:
\vspace{-5mm}
\begin{center}
    \begin{equation*}
        s_L = \frac{\Omega_0^2+s^2}{Bs}
    \end{equation*}
\end{center}

Substituting $\Omega_0 = 0.61945$ and B = 0.573, we get:
\vspace{-5mm}
\begin{center}
    \begin{equation*}
        s_L = \frac{0.3837+s^2}{0.573s}
    \end{equation*}
\end{center}

Substituting this in the low pass transfer function gives us the analog bandpass transfer function. The denominator and numerator are represented by D(s) and N(s) respectively.\\

The coefficients of D(s) are:

\begin{table}[H]
		% Center the table
		\begin{center}
		\begin{tabular}{|c|c|c|c|c|c|c|c|c|c|c|c|c|c|}
			% To create a horizontal line, type \hline
			\hline
			% To end a column type &
			% For a linebreak type \\
			Degree & $s^{14}$  & $s^{13}$ & $s^{12}$ & $s^{11}$ & $s^{10}$ & $s^{9}$ & $s^8$ \\
			
			\hline
                Coefficient  & 1  & 0.4641 & 3.3685 & 1.2997 & 4.5353 & 1.4083 & 3.1431 \\
                \hline
            
		\end{tabular}
		\end{center}
\end{table}

\vspace{-10mm}

\begin{table}[H]
		% Center the table
		\begin{center}
		\begin{tabular}{|c|c|c|c|c|c|c|c|c|c|c|c|c|c|}
			% To create a horizontal line, type \hline
			\hline
			% To end a column type &
			% For a linebreak type \\
			Degree &  $s^7$ & $s^6$  & $s^5$ & $s^4$ & $s^3$ & $s^2$ & $s^1$ & $s^0$ \\
			
			\hline
                Coefficient  & 0.7510 & 1.2061 & 0.2074 & 0.2562 & 0.0282 & 0.0280 & 0.0015 & 0.0012 \\
                \hline
            
		\end{tabular}
		\end{center}
\end{table}

And $N(s) = 5.1107 \times 10^{-4} s^7$



Coefficients of all other powers of s in N(s) are all 0.


\subsection{Discrete Time Filter Transfer Function}

To get the Discrete time transfer function, we need to perform the bilinear transform on the analog  bandpass transfer function.
The bilinear transform is given by:
\vspace{-5mm}
\begin{center}
    \begin{equation*}
        s = \frac{1-z^{-1}}{1+z^{-1}}
    \end{equation*}
\end{center}

Substituting this in the analog bandpass transfer function, we get a rational function of z given by N(z)/D(z).

The coefficients of N(z) are:


\begin{table}[H]
		% Center the table
		\begin{center}
		\begin{tabular}{|c|c|c|c|c|c|c|c|c|c|c|c|c|c|}
			% To create a horizontal line, type \hline
			\hline
			% To end a column type &
			% For a linebreak type \\
			Degree  & $z^{-14}$ & $z^{-12}$ & $z^{-10}$ & $z^{-8}$ & $z^{-6}$  & $z^{-4}$  \\
			
			\hline
                Coefficient  & $-2.8877 \times 10^{-5}$ & $2.0214 \times 10^{-4}$ & $-6.0641 \times 10^{-4}$ & 0.0010 & -0.0010 & $6.0641 \times 10^{-4}$ \\
                \hline
            
		\end{tabular}
		\end{center}
\end{table}

\vspace{-10mm}

\begin{table}[H]
		% Center the table
		\begin{center}
		\begin{tabular}{|c|c|c|c|c|c|c|c|c|c|c|c|c|c|}
			% To create a horizontal line, type \hline
			\hline
			% To end a column type &
			% For a linebreak type \\
			Degree  & $z^{-2}$ & $z^0$  \\
			
			\hline
                Coefficient & $-2.0214 \times 10^{-4}$ & $2.8877 \times 10^{-5}$ \\
                \hline
            
		\end{tabular}
		\end{center}
\end{table}

Coefficients of all odd powers of s in N(s) are all 0.\\


The coefficients of D(z) are:

\begin{table}[H]
		% Center the table
		\begin{center}
		\begin{tabular}{|c|c|c|c|c|c|c|c|c|c|c|c|c|c|}
			% To create a horizontal line, type \hline
			\hline
			% To end a column type &
			% For a linebreak type \\
			Degree & $z^{-14}$  & $z^{-13}$ & $z^{-12}$ & $z^{-11}$ & $z^{-10}$ & $z^{-9}$ & $z^{-8}$ \\
			
			\hline
                Coefficient & 0.5299 & -3.1872 & 11.7694 & -30.4846 & 61.8841 & -101.2390 & 137.8179 \\
                \hline
            
		\end{tabular}
		\end{center}
\end{table}

\vspace{-10mm}

\begin{table}[H]
		% Center the table
		\begin{center}
		\begin{tabular}{|c|c|c|c|c|c|c|c|c|c|c|c|c|c|}
			% To create a horizontal line, type \hline
			\hline
			% To end a column type &
			% For a linebreak type \\
			Degree &  $z^{-7}$ & $z^{-6}$  & $z^{-5}$ & $z^{-4}$ & $z^{-3}$ & $z^{-2}$ & $z^{-1}$ & $z^0$ \\
			
			\hline
                Coefficient & -156.8791 & 150.7805 & -121.2137 & 81.1190 & -43.7856 & 18.5293 & -5.5069 & 1 \\
                \hline
            
		\end{tabular}
		\end{center}
\end{table}

Hence, this, N(z)/D(z) is the final discrete time transfer function of our Chebyschev bandpass filter.

\subsection{Matlab plots}

\begin{figure}[h!]

\centering
\includegraphics[scale = 0.5]{Chebyschev mag phase.png}
\caption{Frequency Response}
\end{figure}

From the above figure, it can be seen that the phase response is not linear.
\clearpage

\begin{figure}[h!]

\centering
\includegraphics[scale = 0.45]{Chebyschev freq res.png}
\caption{Magnitude plot}
\end{figure}

From the above figure, it can be seen that the required specifications are satisfied.

\begin{figure}[h!]

\centering
\includegraphics[scale = 0.7]{Pole zero.png}
\caption{Pole zero plot}
\end{figure}

From the above fig, we can see that all the poles are in the unit circle. Hence the transfer function is stable.

\section{Peer Review of Assignment}

I have reviewed the report of my peer:

\begin{center}
    Annirudh K P\\
    (Roll no: 210070009, Filter number: 96)
\end{center}
and certify it to be correct.



\end{document}
