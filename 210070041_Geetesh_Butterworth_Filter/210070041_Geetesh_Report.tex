\documentclass{article}
\usepackage{graphicx} % Required for inserting images
\usepackage{amsmath}
\usepackage{float}

\title{Filter Design Report}
\author{Name: Geetesh Kini, Roll no: 210070041, Filter no: 104, Group No: 15,\\ Reviewed by Group member: Chanakya Varude, Roll no: 210070092}
\date{February 2023}

\begin{document}

\maketitle

\tableofcontents
\clearpage

\section{Filter Details (Bandstop)}
\subsection{Un-normalized Discrete Time Filter Specifications}

Filter number = 104\\
Since filter number $>$ 80, m = 104 - 80 = 24 and passband will be monotonic.\\
q(m) = greatest integer strictly less than 0.1*m = 2\\
r(m) = m - 10*q(m) = 4\\
$B_L$(m) = 20 + 3*q(m) + 11*r(m) = 20 + 3*2 + 11*4 = 70KHz \\
$B_H$(m) = $B_L$(m) + 40 = 110KHz\\

\vspace{1.5em}
\noindent

This filter is given to be a Bandstop Filter with stopband from BL(m) kHz to BH(m) kHz.
Therefore the specifications are:
\begin{itemize}
    \item Stopband : \textbf{70 - 110 KHz}
    \item  Transition band : \textbf{5KHz} on either side of stopband
    \item Passband : \textbf{0 - 65} and \textbf{115 - 212.5 KHz} (As \textbf{sampling rate} is \textbf{425KHz})

    \item  Tolerance : \textbf{0.15} in \textbf{magnitude} for both passband and stopband
    \item  Nature : Both passband and stopband are \textbf{monotonic}
\end{itemize}


\subsection{Normalized Digital Filter Specifications}
Sampling rate = 425KHz\\
In the normalized frequency axis, sampling rate corresponds to 2$\pi$\\
Therefore, any frequency can be normalized as follows :
\begin{equation*}
    \omega = \frac{\Omega*2\pi}{\Omega_s}
\end{equation*}
where $\Omega_s$ is the Sampling Rate.\\

\vspace{1em}
\noindent
For the normalized discrete filter specifications, the nature and tolerances being the dependent variables remain the same while the passband and stopband frequencies change as per the above transformations. 
\begin{itemize}
    \item Stopband : (\textbf{0.329 -  0.518}) {$\pi$}
    \item  Transition band : \textbf{0.024} $\pi$ on either side of stopband
    \item Passband : (\textbf{0 - 0.306}) {$\pi$} and (\textbf{0.541 - 1}) {$\pi$}
    \item  Tolerance : \textbf{0.15} in \textbf{magnitude} for both passband and stopband
    \item Passband nature: \textbf{Monotonic}
    \item Stopband nature: \textbf{Monotonic}
\end{itemize}

\subsection{Band-stop analog Filter Specifications using Bilinear
Transformation}
To convert to analog domain, we use the following transformation :
\begin{equation*}
    \Omega = \tan (\frac{w}{2})
\end{equation*}
Applying the transformation at Band Edges we get :
\begin{table}[H]
		% Center the table
		\begin{center}
		\begin{tabular}{|c|c|}
			% To create a horizontal line, type \hline
			\hline
			% To end a column type &
			% For a linebreak type \\
			$\omega$ & $\Omega$\\
			
			\hline
                0 & 0\\
                \hline
                0.329 $\pi$ & 0.568 \\
                \hline
                0.518 $\pi$ & 1.058\\
                \hline
                0.306 $\pi$ & 0.521\\
                \hline
                0.541 $\pi$ & 1.137\\
                \hline
                $\pi$ & $\infty$\\
                \hline
            
		\end{tabular}
		\end{center}
\end{table}

Hence, the specifications for corresponding bandstop analog filter are:
\begin{itemize}
    \item Stopband : (\textbf{0.568 -  1.058})
    \item  Transition band : \textbf{0.521 to 0.568} and \textbf{1.058 to 1.137}
    \item Passband : (\textbf{0 - 0.521}) and (\textbf{1.137 - $\infty$})
    \item  Tolerance : \textbf{0.15} in \textbf{magnitude} for both passband and stopband
    \item Passband nature: \textbf{Monotonic}
    \item Stopband nature: \textbf{Monotonic}
\end{itemize}

\subsection{Low pass analog Filter Specifications using Frequency
Transformation}

We need to transform the bandstop filter to a low pass filter. Hence, we can use the bandstop transformation:
\vspace{-5mm}
\begin{center}
    \begin{equation*}
        \Omega_L = \frac{B\Omega}{\Omega_0^2-\Omega^2}
    \end{equation*}
\end{center}

Where $\Omega_0$ and B can be determined by:
We want $\Omega_{P1}$ to map to +1 and $\Omega_{P2}$ to map to -1.\\
Using this, we get:
\begin{center}
    \begin{equation*}
        \Omega_0 = \sqrt{\Omega_{P1}\Omega_{P2}} = \sqrt{0.521 \times 1.137} = 0.76966
    \end{equation*}
    \begin{equation*}
        B = \Omega_{P2} - \Omega_{P1} = 1.137 - 0.521 = 0.616
    \end{equation*}
\end{center}

\begin{table}[H]
		% Center the table
		\begin{center}
		\begin{tabular}{|c|c|}
			% To create a horizontal line, type \hline
			\hline
			% To end a column type &
			% For a linebreak type \\
			$\Omega$ & $\Omega_L$\\
			
			\hline
                $0^+$ & $0^+$\\
                \hline
                0.521($\Omega_{P1}$) & +1($\Omega_{L_{P1}}$) \\
                \hline
                0.568($\Omega_{S1}$) & 1.297$\Omega_{L_{S1}}$)\\
                \hline
                0.76966($\Omega_0^-$) & $\infty$\\
                \hline
                0.76966($\Omega_0^+$) & - $\infty$\\
                \hline
                1.058($\Omega_{S2}$) & -1.236$\Omega_{L_{S2}}$)\\
                \hline
                1.137($\Omega_{P2}$) & -1($\Omega_{L_{P2}}$)\\
                \hline
                $\infty$ & $0^-$\\
                \hline
            
		\end{tabular}
		\end{center}
\end{table}

\subsection{Low pass Analog filter specifications}

\begin{itemize}
    \item Passband Edge : 1 ($\Omega_{L_{P}}$)
    \item Stopband Edge : min($\Omega_{L_{S1}}$, -$\Omega_{L_{S2}}$) = min(1.297, 1.236) = 1.236 ($\Omega_{L_{S}}$)
    \item  Tolerance : \textbf{0.15} in \textbf{magnitude} for both passband and stopband
    \item Passband nature: \textbf{Monotonic}
    \item Stopband nature: \textbf{Monotonic}
\end{itemize}

\subsection{Analog Lowpass Transfer Function}

The analog filter has both passband and stopband monotonic, hence, we need a \textbf{Butterworth} filter. Now, using the given tolerance as 0.15, we can define 2 new quantities $D_1$ and $D_2$:

\begin{center}
    \begin{equation*}
        D_1 = \frac{1}{(1-\delta)^2} - 1 = \frac{1}{0.85^2} - 1 = 0.3841
    \end{equation*}
    \begin{equation*}
        D_2 = \frac{1}{\delta^2} - 1 = \frac{1}{0.15^2} - 1 = 43.44
    \end{equation*}
\end{center}

Hence, the order of the filter N satisfies the inequality:
\begin{center}
    \begin{equation*}
        N \geq \frac{log(\frac{D_2}{D_1})}{2(log(\frac{\Omega_S}{\Omega_P}))} = 11.158
    \end{equation*}
\end{center}

And we want minimum resources to design this filter, hence we use the smallest possible integer value of n that satisfies the above equation, that is \textbf{N = 12}

Also, the cutoff frequency($\Omega_C$) should satisfy the following constraint:

\begin{center}
    \begin{equation*}
        \frac{\Omega_P}{D_1^{\frac{1}{2N}}} \leq \Omega_C \leq \frac{\Omega_S}{D_2^{\frac{1}{2N}}}
    \end{equation*}
    \begin{equation*}
        1.0407 \leq \Omega_C \leq 1.0562
    \end{equation*}
\end{center}

Thus we can choose the value of $\Omega_C$ to be equal to 1.05\\
Now, the poles of the Transfer function are the roots of the equation:
\begin{center}
    \begin{equation*}
        1 + (\frac{s}{j\Omega_C})^{2N} = 1 + (\frac{s}{1.05j})^{2N} = 0
    \end{equation*}
\end{center}

Solving for the roots(using Wolfram) we get:

\begin{figure}[h!]

\centering
\includegraphics[scale = 0.8]{Poles of TF.png}
\caption{Poles of Magnitude Plot of Analog LPF}
\end{figure}

For a stable Filter, we must include the
poles lying in the Left Half Plane in the Transfer Function.\\
$p_1$ = -1.04102 - 0.137053 i\\
$p_2$ = -1.04102 + 0.137053 i\\
$p_3$ = -0.970074 + 0.401818 i\\
$p_4$ = -0.970074 - 0.401818 i\\
$p_5$ = -0.833021 - 0.6392 i\\
$p_6$ = -0.833021 + 0.6392 i\\
$p_7$ = -0.6392 - 0.833021 i\\
$p_8$ = -0.6392 + 0.833021 i\\
$p_9$ = -0.401818 - 0.970074 i\\
$p_{10}$ = -0.401818 + 0.970074 i\\
$p_{11}$ = -0.137053 - 1.04102 i\\
$p_{12}$ = -0.137053 + 1.04102 i\\

Using the above poles which are in the left half plane we can write the Analog Lowpass Transfer Function as:



    

      $   H_{analog,LPF}(s_L) = \frac{(\Omega_C)^N}{\Pi_{i=1}^{12} (s_L-p_i)} $\\
     $ = \frac{1.796}{(s_L^2+1.94s_L+1.1025)(s_L^2+1.666s_L+1.1025)(s_L^2+1.2784s_L+1.1025)(s_L^2+0.8036s_L+1.1025)(s_L^2+0.274s_L+1.1025)(s_L^2+2.082s_L+1.1025)} $\\


    Note that the scaling of the numerator is done in order to obtain a DC gain of 1.

\subsection{Analog Bandstop Transfer Function}

The transformation is given by:
\vspace{-5mm}
\begin{center}
    \begin{equation*}
        s_L = \frac{Bs}{\Omega_0^2+s^2}
    \end{equation*}
\end{center}

Substituting $\Omega_0 = 0.76966$ and B = 0.616, we get:
\vspace{-5mm}
\begin{center}
    \begin{equation*}
        s_L = \frac{0.616s}{0.5924+s^2}
    \end{equation*}
\end{center}

Substituting this in the low pass transfer function gives us the analog bandstop transfer function. The denominator and numerator are represented by D(s) and N(s) respectively.\\

The coefficients of D(s) are:

\begin{table}[H]
		% Center the table
		\begin{center}
		\begin{tabular}{|c|c|c|c|c|c|c|c|c|c|c|c|c|c|}
			% To create a horizontal line, type \hline
			\hline
			% To end a column type &
			% For a linebreak type \\
			Degree & $s^{24}$ & $s^{23}$ & $s^{22}$  & $s^{21}$ & $s^{20}$ & $s^{19}$ & $s^{18}$ & $s^{17}$ & $s^{16}$\\
			
			\hline
                Coefficient & 1 & 4.5041 & 17.2642  & 44.4521 & 99.7834 & 181.2948 & 293.3468 & 407.6367 & 511.0157 \\
                \hline
            
		\end{tabular}
		\end{center}
\end{table}

\begin{table}[H]
		% Center the table
		\begin{center}
		\begin{tabular}{|c|c|c|c|c|c|c|c|c|c|c|c|c|c|}
			% To create a horizontal line, type \hline
			\hline
			% To end a column type &
			% For a linebreak type \\
			Degree &  $s^{15}$ & $s^{14}$  & $s^{13}$ & $s^{12}$ & $s^{11}$ & $s^{10}$ & $s^{9}$ & $s^8$ \\
			
			\hline
                Coefficient  & 564.2816 & 566.1763  & 506.0797 & 412.3625 & 300.3152 & 199.3739 & 117.9155 & 63.3677 \\
                \hline
            
		\end{tabular}
		\end{center}
\end{table}

\begin{table}[H]
		% Center the table
		\begin{center}
		\begin{tabular}{|c|c|c|c|c|c|c|c|c|c|c|c|c|c|}
			% To create a horizontal line, type \hline
			\hline
			% To end a column type &
			% For a linebreak type \\
			Degree &  $s^7$ & $s^6$  & $s^5$ & $s^4$ & $s^3$ & $s^2$ & $s^1$ & $s^0$ \\
			
			\hline
                Coefficient  & 29.9961 & 12.8095 & 4.6978 & 1.5344 & 0.4056 & 0.0935 & 0.0145 & 0.0019 \\
                \hline
            
		\end{tabular}
		\end{center}
\end{table}

Coefficients of N(s) are:

\begin{table}[H]
		% Center the table
		\begin{center}
		\begin{tabular}{|c|c|c|c|c|c|c|c|c|c|c|c|c|c|}
			% To create a horizontal line, type \hline
			\hline
			% To end a column type &
			% For a linebreak type \\
			Degree & $s^{24}$ & $s^{22}$ & $s^{20}$  & $s^{18}$ & $s^{16}$ & $s^{14}$ & $s^{12}$ & $s^{10}$ & $s^{8}$\\
			
			\hline
                Coefficient & 1 & 7.1210 & 23.2413 & 45.9725 & 61.3817 & 58.2796 & 40.3480 & 20.5227 & 7.6115 \\
                \hline
            
		\end{tabular}
		\end{center}
\end{table}

\begin{table}[H]
		% Center the table
		\begin{center}
		\begin{tabular}{|c|c|c|c|c|c|c|c|c|c|c|c|c|c|}
			% To create a horizontal line, type \hline
			\hline
			% To end a column type &
			% For a linebreak type \\
			Degree & $s^6$ & $s^4$ & $s^2$  & $s^0$ \\
			
			\hline
                Coefficient & 2.0075 & 0.3574 & 0.0386 & 0.0019 \\
                \hline
            
		\end{tabular}
		\end{center}
\end{table}

Coefficients of all odd powers of s in N(s) are all 0.

\subsection{Discrete Time Filter Transfer Function}

To get the Discrete time transfer function, we need to perform the bilinear transform on the analog  bandstop transfer function.
The bilinear transform is given by:
\vspace{-5mm}
\begin{center}
    \begin{equation*}
        s = \frac{1-z^{-1}}{1+z^{-1}}
    \end{equation*}
\end{center}

Substituting this in the analog bandstop transfer function, we get a rational function of z given by N(z)/D(z).

The coefficients of N(z) are:

\begin{table}[H]
		% Center the table
		\begin{center}
		\begin{tabular}{|c|c|c|c|c|c|c|c|c|c|c|c|c|c|}
			% To create a horizontal line, type \hline
			\hline
			% To end a column type &
			% For a linebreak type \\
			Degree & $z^{-24}$ & $z^{-23}$ & $z^{-22}$  & $z^{-21}$ & $z^{-20}$ & $z^{-19}$ & $z^{-18}$ & $z^{-17}$ & $z^{-16}$\\
			
			\hline
                Coefficient & 0.0617 & -0.378 & 1.8018  & -5.9632 & 16.7570 & -38.7281 & 78.915 & -139.6387 & 222.0975 \\
                \hline
            
		\end{tabular}
		\end{center}
\end{table}

\begin{table}[H]
		% Center the table
		\begin{center}
		\begin{tabular}{|c|c|c|c|c|c|c|c|c|c|c|c|c|c|}
			% To create a horizontal line, type \hline
			\hline
			% To end a column type &
			% For a linebreak type \\
			Degree &  $z^{-15}$ & $z^{-14}$  & $z^{-13}$ & $z^{-12}$ & $z^{-11}$ & $z^{-10}$ & $z^{-9}$ & $z^{-8}$ \\
			
			\hline
                Coefficient  & -314.1368 & 403.4502 & -465.8035 & 490.5015 & -465.8035 & 403.4502 & -314.1368 & 222.0975 \\
                \hline
            
		\end{tabular}
		\end{center}
\end{table}

\begin{table}[H]
		% Center the table
		\begin{center}
		\begin{tabular}{|c|c|c|c|c|c|c|c|c|c|c|c|c|c|}
			% To create a horizontal line, type \hline
			\hline
			% To end a column type &
			% For a linebreak type \\
			Degree &  $z^{-7}$ & $z^{-6}$  & $z^{-5}$ & $z^{-4}$ & $z^{-3}$ & $z^{-2}$ & $z^{-1}$ & $z^0$ \\
			
			\hline
                Coefficient & -139.6387 & 78.915 & -38.7281 & 16.7570 & -5.9632 & 1.8018 & -0.378 & 0.0617 \\
                \hline
            
		\end{tabular}
		\end{center}
\end{table}

The coefficients of D(z) are:

\begin{table}[H]
		% Center the table
		\begin{center}
		\begin{tabular}{|c|c|c|c|c|c|c|c|c|c|c|c|c|c|}
			% To create a horizontal line, type \hline
			\hline
			% To end a column type &
			% For a linebreak type \\
			Degree & $z^{-24}$ & $z^{-23}$ & $z^{-22}$  & $z^{-21}$ & $z^{-20}$ & $z^{-19}$ & $z^{-18}$ & $z^{-17}$ & $z^{-16}$\\
			
			\hline
                Coefficient & 0.0038 & -0.0286 & 0.1651  & -0.6673 & 2.2861 & -6.4758 & 16.1872 & -35.3039 & 69.344 \\
                \hline
            
		\end{tabular}
		\end{center}
\end{table}

\begin{table}[H]
		% Center the table
		\begin{center}
		\begin{tabular}{|c|c|c|c|c|c|c|c|c|c|c|c|c|c|}
			% To create a horizontal line, type \hline
			\hline
			% To end a column type &
			% For a linebreak type \\
			Degree &  $z^{-15}$ & $z^{-14}$  & $z^{-13}$ & $z^{-12}$ & $z^{-11}$ & $z^{-10}$ & $z^{-9}$ & $z^{-8}$ \\
			
			\hline
                Coefficient  & -121.6909 & 194.4124 & -280.5670 & 370.4089 & -443.3166 & 485.5183 & -480.7720 & 433.7621 \\
                \hline
            
		\end{tabular}
		\end{center}
\end{table}

\begin{table}[H]
		% Center the table
		\begin{center}
		\begin{tabular}{|c|c|c|c|c|c|c|c|c|c|c|c|c|c|}
			% To create a horizontal line, type \hline
			\hline
			% To end a column type &
			% For a linebreak type \\
			Degree &  $z^{-7}$ & $z^{-6}$  & $z^{-5}$ & $z^{-4}$ & $z^{-3}$ & $z^{-2}$ & $z^{-1}$ & $z^0$ \\
			
			\hline
                Coefficient & -350.3115 & 255.0995 & -162.5943 & 91.5297 & -42.8253 & 16.9508 & -4.7432 & 1 \\
                \hline
            
		\end{tabular}
		\end{center}
\end{table}

Hence, this, N(z)/D(z) is the final discrete time transfer function of our butterworth bandstop filter.

\subsection{Matlab plots}

\begin{figure}[h!]

\centering
\includegraphics[scale = 0.6]{Magnitude and phase response.png}
\caption{Frequency Response}
\end{figure}
\clearpage
\begin{figure}[h!]

\centering
\includegraphics[scale = 0.7]{Bandstop.png}
\caption{Magnitude plot}
\end{figure}

From the above figure, it can be seen that the required specifications are satisfied.

\section{Peer Review of Assignment}

I have reviewed the report of my peer:

\begin{center}
    Annirudh K P\\
    (Roll no: 210070009, Filter number: 96)
\end{center}
and certify it to be correct.

\end{document}